\documentclass[a4paper,11pt,onecolumn,twoside]{article}
\usepackage[UTF8,fontset=fandol]{ctex}
\usepackage{fancyhdr}
\usepackage{ctex}
\usepackage{xcolor}
\usepackage{colortbl,booktabs}
\usepackage{listings}
\usepackage{multicol} 
\usepackage{amsmath}
\usepackage{indentfirst}
\usepackage{subfig}
\usepackage{appendix}
\usepackage{amsthm}
\usepackage{amssymb}
\usepackage{makecell}
\usepackage{graphicx}
\usepackage{xcolor}
\usepackage{float}
\usepackage{bm} 
\usepackage{hyperref}
\usepackage{enumitem}
\usepackage[left=1.7cm,right=1.7cm,top=1in,bottom=1in]{geometry}
\theoremstyle{definition} % 定义环境样式为“定义”
\newtheorem{definition}{定义} % 定义 definition 环境
\theoremstyle{plain} % 定理环境样式为“普通”
\newtheorem{theorem}{定理}
\newtheorem{lemma}[theorem]{引理}
\newtheorem{corollary}[theorem]{推论}


\begin{document}

\section{定义}

\subsection{符号说明}
\begin{itemize}[leftmargin=2em]
    \item $\mathcal{W} \subseteq \mathbb{R}^d$: 神经网络参数空间
    \item $W \in \mathcal{W}$: 模型权重向量
    \item $\mathcal{X}$: 输入空间
    \item $\mathcal{Y}$: 输出空间
    \item $M_W: \mathcal{X} \to \mathcal{Y}$: 由$W$参数化的模型
    \item $\mathcal{A}$: 攻击算法
    \item $Q$: 攻击者可用的查询次数上限
    \item $\epsilon, \delta$: 隐私参数
    \item $\sim$: 相邻关系符号
\end{itemize}

\subsection{定义}
\begin{definition}[参数相邻]
设 \( W, W' \in \mathcal{W} \) 为两个神经网络权重向量,若两者的 \( L_2 \)-范数距离满足 \( \|W - W'\|_2 \leq 1 \),则称 \( W \) 与 \( W' \) 是相邻的,即$W \sim W'$。
\end{definition}

\begin{definition}[防御机制]
对于查询 \( x \),防御机制 \( \mathcal{M} \) 返回:
\[
\mathcal{M}(x; W) = f_W(x) + \eta, \quad \eta \sim \mathcal{N}(0, \sigma^2 I_m)
\]
其中 \( f_W(x) \in \mathbb{R}^m \) 为模型在权重 \( W \) 下的Logits输出,\( \eta \) 为各维度独立的高斯噪声。
\end{definition}

\begin{definition}[模型提取攻击]
模型提取攻击是一个概率算法$\mathcal{A}$,给定:
\begin{enumerate}[label=(\arabic*), leftmargin=2em]
    \item 通过查询接口$\mathcal{O}$对目标模型$M_W$的预言机访问
    \item 自适应查询预算$Q \in \mathbb{N}$
\end{enumerate}
输出一个替代模型$\tilde{M}$,使得对于某个相似度度量$\text{sim}$和阈值$\tau$:
\[
\Pr\left[\text{sim}(M_W, \tilde{M}) \geq \tau\right] \geq \beta
\]
其中$\beta \in (0,1]$是成功概率。
\end{definition}

\begin{definition}[模型提取攻击抵抗性]
一个具有查询接口$\mathcal{O}$的模型$M_W$是$(\alpha,\beta)$-抵抗提取攻击的,如果对于任何概率多项式时间攻击算法$\mathcal{A}$,最多对$\mathcal{O}$进行$Q$次查询,都有:
\[
\Pr\left[ \|W_{\text{ext}} - W\|_2 \leq \alpha \right] \leq \beta
\]
其中$W_{\text{ext}}$是提取模型的参数,概率取遍$\mathcal{O}$和$\mathcal{A}$的随机性。
\end{definition}

\begin{definition}[具有差分隐私的查询接口]
一个查询接口$\mathcal{O}_{\epsilon,\delta}$满足$(\epsilon,\delta)$-差分隐私,如果对于所有相邻权重向量$W \sim W'$,所有查询序列$\mathbf{x} = (x_1, \dots, x_n)$,以及所有可测集$S \subseteq \mathcal{Y}^n$:
\[
\Pr[\mathcal{O}_{\epsilon,\delta}(M_W, \mathbf{x}) \in S] \leq e^{\epsilon} \cdot \Pr[\mathcal{O}_{\epsilon,\delta}(M_{W'}, \mathbf{x}) \in S] + \delta
\]
其中概率取遍$\mathcal{O}_{\epsilon,\delta}$的随机性。
\end{definition}

\section{Proof}

\subsection{定义}
攻击者限制访问次数为$Q$,在第$i$次访问中,攻击者向$\mathcal{O}$访问$\{x_i\}$,然后$\mathcal{O}$向攻击者返回$\{y_i=M_W(x_i)+\eta\}$,其中$\eta \sim \mathcal{N}(0, \sigma^2 I_m)$;然后攻击者根据$\{x_i,y_i\}$输出$W_{\text{ext}}$,需要证明模型可以抵抗提取,即证明$\Pr\left[ \|W_{\text{ext}} - W\|_2 < \alpha \right] < \beta$。

\subsection{证明过程}

\begin{theorem}[基于高斯输出扰动的模型提取抵抗性]
设 $M_W:\mathcal X\to\mathbb R^m$ 为一个参数化模型,其参数为
$W\in\mathbb R^d$。假设对任意固定输入 $x\in\mathcal X$,模型输出
关于参数 $W$ 是 $L$-Lipschitz 的,即
\[
\|M_W(x)-M_{W'}(x)\|_2 \le L\|W-W'\|_2,\quad \forall W,W'.
\]
攻击者至多进行 $Q$ 次(可自适应)查询,每次通过如下带噪声的预言机获得输出:
\[
\mathcal O(x)=M_W(x)+\eta,\qquad \eta\sim\mathcal N(0,\sigma^2 I_m),
\]
并最终输出参数估计 $W_{\mathrm{ext}}$。

则对任意 $\alpha>0$,存在常数 $c>0$,使得
\[
\Pr\bigl(\|W_{\mathrm{ext}}-W\|_2< \alpha\bigr)
\;\le\;
\frac{2QL^2 R^2/\sigma^2+\log 2}{c\,d\,log\frac{R}{\alpha}}.
\]
\end{theorem}


\begin{proof}
证明思路是将连续参数估计问题归约为有限多假设判别问题,并应用 Fano 不等式。

\paragraph{1. 参数空间的打包构造}
在 $B(0,R)$ 中构造一个 $2\alpha$-打包集合
\[
\mathcal W=\{W^{(1)},\dots,W^{(N)}\},
\]
满足任意 $i\neq j$ 都有
$\|W^{(i)}-W^{(j)}\|_2\ge 2\alpha$,
且其规模满足 $\log N\ge c\;d\;log\frac{R}{\alpha}$,其中 $c>0$ 为常数。
假设真实参数 $W$ 从集合 $\mathcal W$ 中均匀随机选取,
并用随机变量 $V\in\{1,\dots,N\}$ 表示其索引。


\paragraph{2. 从参数估计到分类问题的归约}
攻击者给出的估计为 $W_{\mathrm{ext}}$。
定义判决规则
\[
\hat V=\arg\min_j \|W_{\mathrm{ext}}-W^{(j)}\|_2.
\]
由打包集合的分离性可知,若
$\|W_{\mathrm{ext}}-W\|_2\le \alpha$,
则必然有 $\hat V=V$。
因此,
\[
\Pr(\|W_{\mathrm{ext}}-W\|_2\le \alpha)
\;\le\;
\Pr(\hat V=V).
\]

\paragraph{3. 观测分布及 KL 散度}
记攻击者观测到的全部输出为
$Y=(y_1,\dots,y_Q)$。
在条件 $V=i$ 下,有
\[
Y \sim \mathcal N(\mu_i,\sigma^2 I_{Qm}),
\quad
\mu_i=(M_{W^{(i)}}(x_1),\dots,M_{W^{(i)}}(x_Q)).
\]
对任意 $i\neq j$,对应分布的 KL 散度为
\[
\mathrm{KL}(P_{W^{(i)}}\|P_{W^{(j)}})
=
\frac{1}{2\sigma^2}
\sum_{q=1}^Q
\|M_{W^{(i)}}(x_q)-M_{W^{(j)}}(x_q)\|_2^2.
\]
由 Lipschitz 条件及
$\|W^{(i)}-W^{(j)}\|_2\le 2R$,可得
\[
\mathrm{KL}(P_{W^{(i)}}\|P_{W^{(j)}})
\le
\frac{2QL^2 R^2}{\sigma^2}.
\]


\paragraph{4. 互信息上界}
设 $P_Y=\frac1N\sum_{j=1}^N P_{W^{(j)}}$ 为 $Y$ 的边缘分布。
由于 $V$ 服从均匀分布,有
\[
I(V;Y)
=
\frac1N\sum_{i=1}^N
\mathrm{KL}(P_{W^{(i)}}\|P_Y).
\]
利用 KL 散度对第二个参数的凸性,可得
\[
I(V;Y)
\le
\frac{1}{N^2}
\sum_{i,j}
\mathrm{KL}(P_{W^{(i)}}\|P_{W^{(j)}})
\le
\frac{2QL^2 R^2}{\sigma^2}.
\]

\paragraph{5. 应用 Fano 不等式}
由 Fano 不等式,
\[
\Pr(\hat V\neq V)
\ge
1-\frac{I(V;Y)+\log 2}{\log N},
\]
等价地,
\[
\Pr(\hat V=V)
\le
\frac{I(V;Y)+\log 2}{\log N}.
\]
结合上述结论即可得到定理结论。
\end{proof}

\end{document}